%%%%%%%%%%%%%%%%%%%%%%%%%%%%%%%%%%%%%%%%%%%%%%%%%%
% 
% TITLE: Oyster Representations Paper Journal
% AUTHOR: Kenneth Fortino
% STARTED: 2024-01-08
%
% DESCRIPTION:
%  This is a journal to develop writing and ideas for the oyster representations paper
%
% CC-BY 
%
% NOTE: For notes on how I got BibTeX to work, see afternoon reflection from 8 January 2024
%%%%%%%%%%%%%%%%%%%%%%%%%%%%%%%%%%%%%%%%%%%%%%%%%%

\documentclass{article}
\usepackage{apacite}
\usepackage{lineno}
\linenumbers

\begin{document}

\title{Journal for Oyster Representations Project}
\author{Kenneth Fortino}
\date{\today}

\maketitle

\section{Ecosystem services and non-human labor}

A key model for conceptualizing the relational structure of oysters within the econo--ecosystem\footnote{The econo--ecosystem highlights the interdependency of human and non--human systems and recognizes that ``nature'' is the product of non--human labor \cite{gibson-graham_non-human_2020}} of the Chesapeake Bay is the ecosystems services model
\footnote{Although the idea the ecosystems provide services to humans is not new, the contemporary idea of ecosystem services was formalized by the United Nations Millennium Ecosystems Assessment which defined ecosystem services as ``benefits people obtain from ecosystems'' and divide these benefits into provisioning services that create the resources humans need, such as food and water, regulating services that maintain environmental variability within ranges of human tolerance, supporting services that create and maintain the biophysical systems that humans depend upon, and cultural services that provide for the intangible benefits humans derive from nature \cite{reid_millennium_2005}}. 
Although in not inherently an economic model, when combined with neoliberal ideas the ecosystem services model allows for the inclusion of the labor of non--human organisms into the economics of the system by modeling the value of the products of this labor if it was performed by humans within the market\footnote{\cite{costanza_value_1998}}. The idea behind the ecosystem services model is that the quantification of market value for the products of non--human labor will incentivize the conservation of the ecosystem that supports that labor  through market forces\footnote{\cite{costanza_value_1998}}. In the case of oysters, researchers have documented a long list of ecosystem services that are provided by oyster reefs, including water quality improvement, shoreline stabilization, and habitat creation\footnote{\cite{grabowski_economic_2012, coen_oyster_date}}. \citeauthor{grabowski_economic_2012} found that the market value of the ecosystem services provide by oyster reefs substantially exceeded the market value of the oysters if they were harvested for meat\footnote{The highest value ecosystem services  provided by the oyster reefs, according to \cite{grabowski_economic_2012} was shoreline protection. When this service was included in the analysis, the reefs recovered their cost of construction within 2 years of construction. However, even when shoreline protection was omitted from the analysis, the reefs recovered their cost of construction within a decade.} 
thereby suggesting that the market should incentivize the creation of sanctuary reefs (i.e., reefs that are not open to harvest). This conceptualization of oyster reefs as the providers of ecosystem services embeds the reefs and the labor of the oysters within the capitalist market system and relies on market forces and capitalist values to define conservation goals. The problem with this approach is that the goals of the capitalist system, to maximize productivity and profit, means that the ecosystem services model does not actually incentivize conservation but in fact incentivises intensification\footnote{need to see \cite{bommarco_ecological_2013}} and the maximization of value through optimization and efficiency increases. 

An example of the co--option of non--human labor for the maximization of production and profit can be seen in the movement to harness the power of soil microbiota to create soil fertility\footnote{\cite{krz_nonhuman_2020}}. There is increasing recognition that the fertility of the soil is the result of the labor of soil microbiota and therefore the productivity of a farm has changed from being ``an activity carried out predominantly by human bodies to an activity carried out by the soil biota under human management''\footnote{\cite[p. ?]{krz_nonhuman_2020}}. The recognition of this ecosystem service (i.e., the creation of soil fertility), by soil biota, has not lead to the conservation of soil ecosystems but has rather lead to the ``direct and indirect manipulation [of the lives of the soil biota] in the name of capital accumulation through e.g. greater efficiency and productivity\ldots''\footnote{\cite[p. 239]{krz_nonhuman_2020}}. Although \citeauthor{krz_nonhuman_2020} do not explicitly reference the ecosystem services model in their example, it is clear that the farmers that they interview see the nonhuman labor of the soil biota primarily through the lens of the services they provide. \citeauthor{krz_nonhuman_2020} notes that for the farmers, ``what matters about agrarian soils\ldots is not so much what they are but what they can do''\footnote{\cite[p. 234]{krz_nonhuman_2020}}. This emphasis on the ``services'' that the soils provide when combined with the goals of a capitalist system --- namely the accumulation of surplus value --- results in a representation of the system that invites reduction. If the system is not a ``system'' per se, but rather an aggregation of ecosystem services, then there is no barrier to the isolation and optimization of those services in the name of production. For the farmers interviewed by {\cite{krz_nonhuman_2020} the primary goal of their ``collaboration'' with the soils was ``the promise of greater farm productivity that soil biota enable''footnote{\cite[p. 243]{krz_nonhuman_2020}}.

An alternative to the ecosystem services model for understanding ecosystems is the foundation species model. In the foundation species model the persistence of an ecosystem is facilitated by one or a few species that create biotic and abiotic habitat for the other species in the system and stabilizes the biogeochemical environment\footnote{cite Foundation species here}. Unlike the ecosystem services model of ecosystems, the foundation species model is not explicitly defined by its relationship with humans or human activities. Humans are incorporated into the ecosystem in relation to the existing structure and processes created by the foundation species. The nature of human relations is not explicitly defined as in the case of the ecosystem services model, where benefits flow from nature to humans\footnote{\cite{costanza_value_1998}}. Because foundation species are strong interactors
\footnote{Strong interactors are species that have an impact on the structure or function of an ecosystem that is disproportionate to either their abundance or the impact of other species in the system [CITE strong interactors here]}
human--caused alterations to their abundance or function will have disproportionately large effects on the ecosystem. Hemlock forests create a unique physiochemical environment due to the impacts of their leaf litter on soil nutrient content, soil moisture, and light availability, that supports a unique community of facilitated organisms\footnote{\cite{ellison_2005}}. However, hemlock forests do not reestablish themselves following harvest by humans but are replaced by hardwood species\footnote{\cite{ellison_2005}}, so the exploitation of hemlock trees as a raw material results in not only the co--opting of the metabolic labor of the hemlock trees but undermines their creative power within the system. The application of the ecosystem services model to this system would recognize that in addition to the market value of the wood provided by the forest, the hemlock forest might also provide services that are valuable to humans, such as recreation, habitat for other valuable species, or a repository of bio--products such as medicine
As a result, the total value of the forest to humans could exceed the market value of the wood and the market should drive its preservation\footnote{\cite{costanza_value_1998}}. Although under this analysis, the forest may be preserved, the forest ecosystem has been reduced to simply a spreadsheet of services. \citeauthor{battistoni_bringing_2017} writes\footnote{\cite[p. 11]{battistoni_bringing_2017}}

\begin{quote}
Turning ecosystems into property requires that they be represented for the market as an array of individualized services that fails to adequately reflect their actual functioning or necessary independence; thus the complexity and relationality of what is being preserved is often lost as ecosystems are divided into packages of services\ldots
\end{quote}

The ``complexity and relationality'' of the ecosystem that \citeauthor{battistoni_bringing_2017} refers to here is precisely what is created by the foundation species. It is through the relationships with the other species in the system that the foundation species ``creates'' a unique ecosystem. In this sense the ecosystem is not an aggregation of services but the result of the emergent properties of organisms in relation.

The foundation species concept shows that the ecosystem that emerges from the labor of the foundation species is more than just a representation applied by humans but has biological materiality. The ecosystem is a ``thing'' that is created by the relational structure and emergent properties of its constituents in collaboration with the labor of the foundation species. Therefore, as \citeauthor{battistoni_bringing_2017} notes, the ecosystem has ``necessary independence'' as well. Through the recognition of the ecosystem's materiality and creative agency, the ecosystem becomes not only economic but also political. That is to say that the ecosystem and it's members are represented not by the value that they bring to the market but as co--creators, as \citeauthor{battistoni_bringing_2017} says ``as a collective distributed undertaking of humans and nonhumans to reproduce, regenerate, and renew a common world'' through ``hybrid--labor''\footnote{\cite[p. 6]{battistoni_bringing_2017}}. This idea is also represented by extending Marx's concept of ``species--being'' to nonhumans, where nonhumans as well as humans labor within a relational framework with others for their own wellbeing\footnote{\cite{fair_toward_2023}}. 

Any attempt at conservation risks creating a distinction between the ``natural'' and the ``human'', and then seeking to erase the ``human'' from the ``natural'' to return to a preferred ``pristine'' state. \citeauthor{latour_we_1993} classically showed that the distinction between the nature and culture is a myth of modernity but nonetheless it remains a compelling and persistent model influencing our interactions with the environment. For our present analysis, it becomes relevant in the application of the ecosystem services model to conservation. The ecosystem services model is ``the idea that we should care for the non--human world because of all the services it provides to humans to maintain the world we need and want'' \footnote{\cite{gibson-graham_non-human_2020}}. In this conception the needs and wants of humans seen as distinct from the needs and wants of the non--human and therefore permits the exploitation of non--human labor to serve the needs and wants of humans. However, this model fails to recognize the interdependency of the human and non--human worlds for the co--creation of ``nature'' \footnote{\cite{gibson-graham_non-human_2020, richardson_introduction_2014, krz_nonhuman_2020}}. \citeauthor{gibson-graham_non-human_2020} propose the concept of the ``econo--ecological'' system, which highlights the interdependency of human and non--human interactions. This model alludes to the same relational structure that ecologists have recognized in the foundation species model, where the structure and function of the system is the result of facilitating interactions between its members, what \citeauthor{gibson-graham_non-human_2020} call ``in--kind'' labor interactions. In this type of relational structure, the labor of one species provides the conditions necessary for other species to thrive. 

In my attempt to put ecological models and humanist frameworks in conversation, I am coming up against an issue of ``translation'' between the two fields. The humanist emphasis on ``thriving'' and ``well--being'' that is captured in \citeauthor{fair_toward_2023} description of ``species--being'', whereby a species is capable of applying its labor for its own welfare, does not map well to ecological understandings of the success for species, which are rooted in the Darwinian idea of ``fitness''. From the perspective of biological evolution, fitness is the number of reproductively mature offspring that an individual produces. So if one squirrel individual produces 4 reproductively viable offspring, and another squirrel individual produces only 3 reproductively viable offspring, then the former is considered to have greater fitness. This matters, of course, because the principle metric of ``success'' in evolutionary biology is the temporal transference of genetic information, which is done through the production of viable offspring. This narrow definition of ``success'' is not easily reconciled with humanist ideas of ``thriving'' or ``well--being'', since they produce a conundrum whereby we can recognize that from a human perspective these terms do not simply mean the production of viable offspring, and in the case of a feminist ideas\footnote{\cite{}}, may explicitly reject reproduction as a definition of well--being. On the other hand, since it is impossible to know the experience of non--humans\footnote{\cite{fair_toward_2023}}, applying human--based definitions of thriving or well--being to non--humans is irrevocably fraught. For the purposes of this project then, I an drawn to the idea that ``thriving'' and ``well--being'' are connected to being able to participate in the full suite of ecological relationships that reflect a species' evolutionary history. 

Evolutionary history reflect the synthesis of relational structure and creation, since changes in the structural and genetic information of a species
\footnote{Structural information refers to the specific arrangement of materials that make up an individual of a species. The structural information is created through the process of development by the genetic information, which is the specific sequence of nucleotide bases in the DNA molecules of the individual's genome. The two forms of information in inextricably linked in that the structural information (i.e., the biological configuration of the organism) is needed to \emph{use} the genetic information and the genetic information \emph{specifies} the structural information, mostly by specifying which enzymes the individual can synthesize.}
are linked to environmental (i.e., relational) factors that an individual encounters. This transfer of information and matter through time by the combined processes of biological evolution and metabolism aligns with the connections between matter and semiotics in new materialist ideas\footnote{\cite{iovino_theorizing_2012}} where matter and meaning are entangled and arise from the ``world's process of becoming''\footnote{\cite[p. 453]{iovino_theorizing_2012}}. This act of co--creation of biomass and information, matter and semiotics, linked through a temporally specific relational structure represents in the most explicit way, what it means to be an individual and a species. \citeauthor{tsing_mushroom_2021} describes the process of ``alienation'' as being removed from the context in which the developed or exist. The specific ecological relational structure of a species and the history that created it most basically this context of development and existence. Therefore, alienation from the specific ecological relationships that reflect a species' evolutionary history represents alienation from ``thriving'' and a species capacity to labor toward its own ``well--being'', that is to manifest its species--being\footnote{\cite{fair_toward_2023}}, irrespective of whether a species is producing viable offspring. This relational structure is why exploitation for ecosystem services, even if the species is productive (or even freed from suffering, as is the goal of much animal liberation activity) is alienation, in that it is impossible to simultaneously maximize a subset of organismal or ecological functions that produce ``services'' for humans and maintain the historically contingent relational structure of the system. The foundation species model, in contrast, inherently recognizes the eco--historical relational structure of the system because what is considered the ``system'' is explicitly the manifestation of those relationships over time\footnote{\cite{angelini_interactions_2011}}.

Emerging ideas about the relationship between matter and meaning can inform our understanding of this eco--historical relational system, that is what it means to develop into an oyster reef in relationship within the context of the Bay. The recognition that non--humans and even inanimate matter can have agentic capacity\footnote{\cite{iovino_theorizing_2012}} means that the development of the reef and its associated ecology has relational, historical, and contingent components. That is to say that the specific interactions that have and are taking place between the living and non--living components of the system are the process of continually creating the reef. Why this matters is because if we are to assert that the conceptualization of ecosystems based on their capacity to provide ecosystem services is an act of alienation in the sense described by \citeauthor{tsing_mushroom_2021}, then we need to have some way of understanding from what the organisms in the system are being alienated. \citeauthor{iovino_theorizing_2012} in their ``Diptych'' on new and postmodern materialism, show how these movements allow us to consider the integration of matter and meaning. In these concepts the relationships between the human, non--human, and even inanimate matter create ``things which are material, specific, non--self--identical, and semiotically active\footnote{\cite[p. 462]{iovino_theorizing_2012}} and thus matter and meaning are entangled and emerge due to ``world's process of becoming''\footnote{\cite[p. 453]{iovino_theorizing_2012}}. Applying these ideas to our oyster reef, we can thus see the oyster reef as the continual creation of the thing that is the reef. There is no reef independent of the eco--historical relationships that are the material and semiotic processes of the combined agentic capacity of the matter of the system. In other words, what exists now as a reef is the product of a process that can be understood narratively and emerges from the meaning--matter interrelationships. Pushing these ideas even further away from the dualism of matter \emph{and} meaning, \citeauthor{barad_paper} argues that agency does not exist independent of the relationships of the actors but rather ``emerges through \emph{intra--action}''\footnote{\cite[p. 466]{iovino_theorizing_2012}}. That is to say that the system does not consist of the coming together of the independent agentic capacity of things but rather the coming together, the \emph{intra--action}, is the agentic capacity. In this conception then, we can see that viewing an ecosystem as an aggregation of ecosystem services completely overlooks processes from which those services arise. The ecosystem services model maintains the dualism of a human that can receive and benefit from services produced by a system that has an independent existence and furthermore that can be acted upon (e.g., certain services maximized for human benefit), independent of the history and relational structure that brought those services into being.
When we see the system as one in which humans benefit from the services of a non--human system, we have rendered the outcomes of non--human labor as resources. That is to say that the ecosystem has become a source of resources for humans to extract. We can connect resources and services if we view resources as ``ubiquitous and energentic substances that play an active part in the making of worlds''\footnote{\cite[p. 6]{richardson_introduction_2014}}. Imagining an oyster reef as a source of resources, immediately draws one to the idea of what can be extracted and removed for use by humans. The most obvious example of this is of course the meat of the oyster, but also the shell, which can be used for building material\footnote{\cite{}} and as a substrate to cultivate more oysters\footnote{\cite{}}. In the case of ecosystem services, the resource that is ``taken'' from the reef is less obvious. When we conceptualize a reef as a provider of services like phytoplankton removal or shoreline protection, are we utilizing the reef as a resource? I argue yes. In this case what is being extracted from the reef is the labor of the oysters. The oysters are no longer seen as contributing labor for their own well being but rather for the benefit of humans\footnote{This is similar to what \citeA{fair_toward_2023} observe for the labor of soil biota.} who are outside of the relational structure of the reef. That is, the resource flows unidirectionally from the oysters to humans and the labor of the oysters is a ``free gift'' of nature that is available for human use\footnote{\cite{richardson_introduction_2014, battistoni_bringing_2017}}. This view however, ignores the relational structure of natural systems where the labor of the oysters are not only in relation with humans but also with myriad other organisms in the complex ecosystem of the reef. The importance of the interdependency of the human and non--human members of the system is highlighted if we consider some of the other ecosystem services attributed to oyster reefs\footnote{For a full list of the ecosystem services that are provided by oyster reefs see \cite{grabowski_economic_2012} and \cite{}}, for example the removal of nitrogen via denitrification. In this case the oysters themselves are incapable of performing denitrification but rather they create the conditions required for denitrifying bacteria to perform this function
\footnote{Denitrifying bacteria are either facultative or obligate anerobes that use nitrate (NO\textsubscript{3}) as a the final electron acceptor in cellular respiration (i.e., the cellular process that extracts biologically useful energy from organic matter). Since nitrate yields less energy than oxygen, denitricication is only favored when there is no oxygen available. The conditions then that favor denitrification are those where oxygen is limited, nitrate is abundant, and organic matter is abundant. It is precisely these conditions that are created in the sediments around oyster reefs \cite{smyth_assessing_2013}}
and therefore are part of a facilitation cascade
\footnote{Facilitation cascades occur when the activity of one species creates the conditions that make the colonization and persistence of another species possible \cite{angelini_interactions_2011}}
that only emerges from the relational structure of the reef. What then is the resource that is being utilized by humans? In the case of denitrification, it is the labor of the denitrifying bacteria that is providing the service that is of use to humans but this labor is only made possible by the world--building activity of the oysters. Similar relationally--dependent structures are observed with other ecosystem services (i.e., resources) provided by oyster reefs, such as habitat creation for economically important fish species\footnote{\cite{}}. It would seem then that the principle resource provided by the oysters then is their capacity to enter into ecological relationships. \citeauthor{richardson_introduction_2014} argue that resources come into being via abstraction which is the ``separation, parting, simplification, and reduction\ldots of both material and conceptual levels.''\footnote{\cite[p. 13]{richardson_introduction_2014}}. When oyster reefs are modeled as a set of ecosystem services, which is to conceive them primarily as resources, the necessary abstraction that is required in the process of resource--making obscures the relational structure that creates the material thing from which humans derive value.



\bibliographystyle{apacite}
\bibliography{Oyster_env_humanities}
\end{document}
