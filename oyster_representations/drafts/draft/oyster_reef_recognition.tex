%%%%%%%%%%%%%%%%%%%%%%%%%%%%%%%%%%%%%%%%%%%%%%%%%%
% 
% TITLE: Oyster Reef Recognition Paper Draft
% AUTHOR: Kenneth Fortino
% COLLABORATOR: Sean Barry
% STARTED: 2024-02-01
%
% DESCRIPTION:
%  A draft of the Oyster Reef Recognition Paper
%
% CC-BY 
%
%%%%%%%%%%%%%%%%%%%%%%%%%%%%%%%%%%%%%%%%%%%%%%%%%%

\documentclass{article}
\usepackage{apacite}
\usepackage{lineno}
\linenumbers

\begin{document}

\title{Non--human labor and oyster reefs}
\author{Kenneth Fortino}
\date{\today}

\maketitle

\section{Oysters in the Chesapeake Bay}

\section{The ecosystem services model is incompatible with the foundation species model of ecosystems}

Ecologists use models to manage the contingency and complexity of ecological systems\footnote{\cite{lawton_laws_1999}} and to be able to predict how ecological systems will behave in relation to human interactions with them\footnote{cite something about resource management here}. The two main ecological models applied to oyster reefs are the \emph{ecosystem services model} and the \emph{foundation species model}. Although these models are frequently references in the same paper\footnote{For examples see: cite BUNCH OF EXAMPLES}, these two models conceptualize ecosystems in fundamentally different and mutually incompatible ways. Therefore, it is impossible to meaningfully apply both models simultaneously to a system, and as I illustrate below, when this is attempted, the fundamental characteristics of the foundation species model is disregarded in favor of the characterizations of the ecosystem services model.

Humans have recognized the benefits to human well--being provided by ecosystems for much of our history\footnote{CITE somthing here}}. However until the twentieth century the benefits provided by nature were identified primarily as ``free gifts''\footnote{See \cite{battistoni_bringing_2017}}

\section{The ecosystem services model alienates the non--human labor of the oyster reef}

\section{The foundation species model \emph{recognizes} the labor and agency of non--humans in political and economic systems}

\section{Conclusion}


\bibliographystyle{apacite}
\bibliography{../Oyster_env_humanities}
\end{document}
