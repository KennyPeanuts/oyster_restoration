%%%%%%%%%%%%%%%%%%%%%%%%%%%%%%%%%%%%%%%%%%%%%%%%%%
% 
% TITLE: Oyster Reef Recognition Paper Draft
% AUTHOR: Kenneth Fortino
% COLLABORATOR: Sean Barry
% STARTED: 2024-02-01
%
% DESCRIPTION:
%  A draft of the Oyster Reef Recognition Paper
%
% CC-BY 
%
%%%%%%%%%%%%%%%%%%%%%%%%%%%%%%%%%%%%%%%%%%%%%%%%%%

\documentclass{article}
\usepackage{apacite}
\usepackage{lineno}
\linenumbers

\begin{document}

\title{Non--human labor and oyster reefs}
\author{Kenneth Fortino}
\date{\today}

\maketitle

\section{Oysters in the Chesapeake Bay}

\section{The ecosystem services model is incompatible with the foundation species model of ecosystems}

Ecologists use models to manage the contingency and complexity of ecological systems\footnote{\cite{lawton_laws_1999}} and to be able to predict how ecological systems will behave in relation to human interactions with them\footnote{cite something about resource management here}. The two main ecological models applied to oyster reefs are the \emph{ecosystem services model} and the \emph{foundation species model}. Although these models are frequently references in the same paper\footnote{For examples see: cite BUNCH OF EXAMPLES}, these two models conceptualize ecosystems in fundamentally different and mutually incompatible ways. Therefore, it is impossible to meaningfully apply both models simultaneously to a system, and as I illustrate below, when this is attempted, the fundamental characteristics of the foundation species model is disregarded in favor of the characterizations of the ecosystem services model.

Humans have recognized the benefits to human well--being provided by ecosystems for much of our history\footnote{CITE something here}. However until the twentieth century the benefits provided by nature were identified primarily as ``free gifts''\footnote{See \cite{battistoni_bringing_2017}} that were not recognized by the economy. With the recognition of the damage that human activities were doing to the ability of nature to continue to sustain human life, there was a movement to incorporate the ``services'' that nature provides into the economic system, recognizing that ``the
human species, while buffered against environmental changes by culture and technology, is fundamentally dependent
on the flow of ecosystem services''\footnote{\cite[p. V]{reid_millennium_2005}}. This recognition is partially accomplished by the calculating what it would cost if the economy would have to ``pay'' nature for the services it provides. Highlighting the magnitude of the contribution of ecosystem services to the \emph{economy}, \citeauthor{costanza_value_1998} concluded that the total value of the ecosystem services provided by the biosphere is at a minimum, US\$33 trillion, which the authors point out is 1.8 times greater than the global GNP in 1998. The implication of \citeauthor{costanza_value_1998} is that the magnitude of the ``free gifts'' of nature, defies compensation even if there was the will. More practically, the calculation of ecosystem services is used to provide a market incentive for the preservation of ecosystems\footnote{\cite{gibson-graham_non-human_2020}}. For example, \citeauthor{grabowski_economic_2012} shows that the value of the services provided by oyster reefs exceeds the value of the harvested oysters
\footnote{The highest value ecosystem services  provided by the oyster reefs, according to \cite{grabowski_economic_2012} was shoreline protection. When this service was included in the analysis, the reefs recovered their cost of construction within 2 years of construction. However, even when shoreline protection was omitted from the analysis, the reefs recovered their cost of construction within a decade.}
thereby suggesting that the market should de--incentivize the harvesting of the reefs. That this has not happened
\footnote{While the production of ``sanctuary'', that is un--harvested reefs has increased since \citeauthor{grabowski_economic_2012} analysis, the harvest of oysters from the Chesapeake Bay has not only not stopped but has increased since the beginning of the twentieth century. In 2000 the total landings of eastern oysters in Maryland and Virginia was 1,148 metric tons. The landings were lowest in 2004 with only 39 metric tons, but since then have climbed to the highest recorded landings in 2022 of 2,811 metric tons (https://www.fisheries.noaa.gov/foss/f?p=215:200:10909893918708:::::). Furthermore this increase in wild harvested oysters is accompanied by an even greater growth in oyster aquaculture, which occupies Bay bottom to the exclusion of natural reefs (CITE).}
raises questions about whether the ecosystems services model is accurately represented in the economic analysis.



\section{The ecosystem services model alienates the non--human labor of the oyster reef}

\section{The foundation species model \emph{recognizes} the labor and agency of non--humans in political and economic systems}

\section{Conclusion}


\bibliographystyle{apacite}
\bibliography{../Oyster_env_humanities}
\end{document}
