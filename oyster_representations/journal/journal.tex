%%%%%%%%%%%%%%%%%%%%%%%%%%%%%%%%%%%%%%%%%%%%%%%%%%
% 
% TITLE: Oyster Representations Paper Journal
% AUTHOR: Kenneth Fortino
% STARTED: 2024-01-08
%
% DESCRIPTION:
%  This is a journal to develop writing and ideas for the oyster representations paper
%
% CC-BY 
%
% NOTE: For notes on how I got BibTeX to work, see afternoon reflection from 8 January 2024
%%%%%%%%%%%%%%%%%%%%%%%%%%%%%%%%%%%%%%%%%%%%%%%%%%

\documentclass{article}
\usepackage{natbib}

\begin{document}

\title{Journal for Oyster Representations Project}
\author{Kenneth Fortino}
\date{\today}

\maketitle

\section{Ecosystem services and non-human labor}

A key model for conceptualizing the relational structure of oysters within the socio--ecosystem of the Chesapeake Bay is the ecosystems services model. In this model the labor of non--human organisms is included in the economics of the system by modeling the value of the products of this labor if it was performed by humans within the market \footnote{\cite{costanza_value_1998}}. The idea behind the ecosystem services model is that the quantification of market value for the products of non--human labor will incentivize the conservation of the ecosystem that supports that labor  through market forces \footnote{\cite{costanza_value_1998}}. In the case of oysters, researchers have documented a long list of ecosystem services that are provided by oyster reefs, including water quality improvement, shoreline stabilization, and habitat creation \footnote{\cite{grabowski_economic_2012, coen_oyster_date}}. \citet{grabowski_economic_2012} found that the market value of the ecosystem services provide by oyster reefs substantially exceeded the market value of the oysters if they were harvested for meat 
\footnote{The highest value ecosystem services  provided by the oyster reefs, according to \cite{grabowski_economic_2012} was shoreline protection. When this service was included in the analysis, the reefs recovered their cost of construction within 2 years of construction. However, even when shoreline protection was omitted from the analysis, the reefs recovered their cost of construction within a decade.} 
thereby suggesting that the market should incentivize the creation of sanctuary reefs (i.e., reefs that are not open to harvest). This conceptualization of oyster reefs as the providers of ecosystem services embeds the reefs and the labor of the oysters within the capitalist market system and relies on market forces and capitalist values to define conservation goals. The problem with this approach is that the goals of the capitalist system, to maximize productivity and profit, means that the ecosystem services model does not actually incentivize conservation but in fact incentivises intensification \footnote{need to see \cite{bommarco_ecological_2013}} and the maximization of value through optimization and efficiency increases. 

An example of the co--option of non--human labor for the maximization of production and profit can be seen in the movement to harness the power of soil microbiota to create soil fertility \footnote{\cite{krz_nonhuman_2020}}. There is increasing recognition that the fertility of the soil is the result of the labor of soil microbiota and therefore the productivity of a farm has changed from being "an activity carried out predominantly by human bodies to an activity carried out by the soil biota under human management"\footnote{\cite{krz_nonhuman_2020}}. The recognition of this ecosystem service (i.e., the creation of soil fertility), by soil biota, has not lead to the conservation of soil ecosystems but has rather lead to the "direct and indirect manipulation [of the lives of the soil biota] in the name of capital accumulation through e.g. greater efficiency and productivity\ldots"\footnote{p. 239 \cite{krz_nonhuman_2020}}. Although \citet{krz_nonhuman_2020} do not explicitly reference the ecosystem services model in their example, it is clear that the farmers that they interview see the nonhuman labor of the soil biota primarily through the lens of the services they provide. \citet{krz_nonhuman_2020} notes that for the farmers, "what matters about agrarian soils\ldots is not so much what they are but what they can do" \footnote{p. 234, \cite{krz_nonhuman_2020}}. This emphasis on the "services" that the soils provide when combined with the goals of a capitalist system --- namely the accumulation of surplus value --- results in a representation of the system that invites reduction. If the system is not a "system" per se, but rather an aggregation of ecosystem services, then there is no barrier to the isolation and optimization of those services in the name of production. For the farmers interviewed by {\citet{krz_nonhuman_2020} the primary goal of their "collaboration" with the soils was "the promise of greater farm productivity that soil biota enable"\footnote{p. 243 \cite{krz_nonhuman_2020}}.

\bibliographystyle{plain}
\bibliography{Oyster_env_humanities}
\end{document}
