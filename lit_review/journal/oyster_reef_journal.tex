%%%%%%%%%%%%%%%%%%%%%%%%%%%%%%%%%%%%%%%%%%%%%%%%%%
% 
% TITLE: Oyster Reef Literature Review Journal
% AUTHOR: Kenneth Fortino
% STARTED: 2024-01-10
%
% DESCRIPTION:
%  This is a journal to develop writing and ideas for the oyster reef restoration literature review
%
% CC-BY 
%
% NOTE: For notes on how I got BibTeX to work, see afternoon reflection from 8 January 2024
%%%%%%%%%%%%%%%%%%%%%%%%%%%%%%%%%%%%%%%%%%%%%%%%%%

\documentclass{article}
\usepackage{apacite}

\begin{document}

\title{Journal for Oyster Reef Restoration Literature Review}
\author{Kenneth Fortino}
\date{\today}

\maketitle

\section{Summary of Ecosystem Services Model}

\section{Summary of Foundation Species Model}

\section{Application of Models to Oyster Restoration and Aquaculture}

Oysters and oyster reefs are conceptualized using both the ecosystem services model and the foundation species model --- sometimes in the same paper \cite{mercaldo-allen_oyster_2023}. There is evidence in the literature to support the use of both of these models for understanding oyster ecology. The foundation species model proposes that an ecosystem contains one or a few species that have a disproportional effect on community structure by altering the physical, chemical, and biological processes of an ecosystem to facilitate and stabilize a specific community \cite{fields_foundation_2022}. The presence of the foundation species creates and stabilizes the physical environment by minimizing fluctuations in temperature, moisture, pH, or other physical parameters \cite{ellison_loss_2005}. Foundation species can also alter ecosystem production through changes in nutrient cycling, in some cases increasing \cite{fields_foundation_2022} or decreasing \cite{ellison_loss_2005} nutrient availability. In some cases, the main effect of the foundation species is to literally build the three--dimensional habitat that other species require \cite{mercaldo-allen_oyster_2023}. Because they are capable of building large biogenic structures that can persist for long time periods, sometimes centuries \cite{lockwood_conservation_2019} and of processing large volumes of materials via filtration \cite{newell_ecological_1988} oyster reefs can be conceptualized as foundation species where they occur. For example, \citeA{newell_ecological_1988} reports that prior to their depletion during the 19th and 20th centuries, oyster reefs could remove 22 --- 44\% of phytoplankton C production in the Chesapeake Bay, indicating a significant alteration of material cycling in the system \cite{fields_foundation_2022}. Other studies support the idea that the function of oyster reefs is dependent on their structural development indicating that it is the creation of a particular habitat and its associated conditions that drives the effect. \citeA{searles_oyster_2022} found that macroinvertebrate communities recovered on the interior of restored reefs but not their margins, suggesting that a certain oyster density or reef structure was required to facilitate the colonization of the macroinvertebrates. The alteration of the physical environment is a key factor in the way that oyster reefs create the specific ecological communities associated with them \cite{lenihan_physical_1999, searles_oyster_2022}. Because the effects of the reef on the structure and function of the reef community are the result of emergent effects that result from the ``physical--biological coupling'' if the reef and the estuarine environment, the deconstruction of the reef by harvesting practices that destroy the three--dimensional characteristics of the reef, as well as remove individuals will result in the deminishment or elimination of the facultative properties of the reef \cite{lenihan_physical_1999}. In this way the reef as a foundation species is defined by the emergent properties that arise from the integrated function of the oysters and the other organisms that are facilitated by the effects that the oysters have on the physical--biological coupling of the system \cite{lenihan_physical_1999, ellison_loss_2005, angelini_interactions_2011}. 



\section{Impact of Models on Understanding Oyster Management}






\bibliographystyle{apacite}
\bibliography{Oyster_Lit_Review}
\end{document}
