%%%%%%%%%%%%%%%%%%%%%%%%%%%%%%%%%%%%%%%%%%%%%%%%%%
% 
% TITLE: Literature review on the use of models for oyster reefs
% AUTHOR: Kenneth Fortino
% STARTED: 2024-02-12
%
% DESCRIPTION:
%  This is the manuscript for submission to peerJ. 
%  The manuscript is based on the guidelines in the "Literature review template" available at: https://peerj.com/about/author-instructions/
%  The LaTeX is formatted according to the specification in the "LaTeX template via Overleaf" available at: https://peerj.com/about/author-instructions/
%  The LaTeX template was used to create a shell document in the `fortinok` Overleaf projects.
%
% CC-BY 
%
% NOTE: For notes on how I got BibTeX to work, see afternoon reflection from 8 January 2024
%%%%%%%%%%%%%%%%%%%%%%%%%%%%%%%%%%%%%%%%%%%%%%%%%%

\documentclass{article}
\usepackage{apacite}


%\begin{abstract}
%\end{abstract}

\begin{document}

\title{Ecological models of oyster reefs affect conceptualization of reefs and conservation goals.}
\author{Kenneth Fortino}
\date{\today}

\maketitle

\section*{Introduction}
\label{sec:intro}

% Add something about how models are used to conceptualize systems in ecology
Oysters and oyster reefs are conceptualized using both the ecosystem services model and the foundation species model---sometimes in the same paper \cite{mercaldo-allen_oyster_2023}. There is evidence in the literature to support the use of both of these models for understanding oyster ecology but the different models lead to different conceptualizations of the system. 

\section*{Survey Methodology}
\label{sec:surv_method}
% Describe the process by which you ensured that your coverage of the literature was comprehensive and unbiased.

This literature review is intended to address the question of how the foundation species model and the ecosystem services model have been applied to understanding oyster reefs and how the application of these models affect the way that we approach oyster reef conservation, restoration, and aquaculture, particularly in the Chesapeake Bay. To compare the ecosystem services model to the foundation species model, I conducted a Web--of--Science search using the key words "ecosystem services" and "foundation species". The study of foundation species is an active and developing area of research and therefore this broad search yielded 1,136 articles. Since the goal of this search was to develop a comprehensive (but not necessarily exhaustive) understanding of the conceptual structure of the foundation species model, I reviewed the first 100 articles of the search, sorted by "Relevance" in the Web--of--Science results. After reading the title and abstracts of these articles, I eliminated any articles that I determined to be focused on the ecology of specific systems and did not contribute to the broader conceptual understanding of foundation species. Following this culling, I was left with 40 articles. These articles were then augmented with relevant articles that were identified from articles cited from the original search. The final number of articles considered for understanding foundation species was NUMBER.

The same approach was applied to understanding of the conceptual structure of the ecosystem services model. Since the concept of ecosystem services spans more disciplines and has received broader academic attention, this initial search yielded 44,748 articles. Since there is a journal titles "Ecosystem Services", this initial search selected only articles from this journal in the list of most relevant articles. To account for this, I conducted a search that omitted this journal from consideration and then I searched this journal specifically. The search that omitted the Ecosystems Services journal yielded 43,233 articles Since there is a journal titles "Ecosystem Services", this initial search selected only articles from this journal in the list of most relevant articles. To account for this, I conducted a search that omitted this journal from consideration and then I searched this journal specifically. The search that omitted the Ecosystems Services journal yielded 43,233 articles. The search of only the Ecosystem Services journal yielded 1,515 articles.  

As with the foundation species model, I again searched the titles and abstracts of the first 100 articles from both searched sorted by relevance in Web--of--Science. In this case I was specifically filtering articles for their capacity to contribute to the understanding of ecosystem services as a conceptual framework for ecological systems and thus, I removed articles that were focused on legal, political, or governmental frameworks, specific to regions or countries, or focused on specific industries or economic sectors. For both searches this limiting yielded 48 articles and these articles were augmented with relevant articles located via citations from other articles.

To identify the way that these models have been applied to oyster reef ecosystems, I repeated the above searches but added "oyster*" as a keyword. The search for "ecosystem services" AND "oysters yielded 389 results. I again went through the first 100 papers sorted by relevance and using the titles and abstracts I filtered these results to select papers that focused on the ecological application of the ecosystem services model to oyster ecosystems and were not overly focused on a specific system. This filtering yielded 41 results.

To identify the way that \emph{Crassostrea virginica} the foundation species model I searched for "c* virginica" and "reef*" to get studies that focused on the ecological interactions of the oysters, not their physiology or genetics. This search yielded 620 results. I filtered these by reviewing the first 100 papers sorted by relevance and selecting papers that focused on wild or restored oyster reef systems (i.e., not aquaculture), were not narrowly focused on oyster physiology or single trait biology. This yielded 38 references. I added references also from the search of "oyster*" AND "Chesapeake". 

% [This approach did not actually produce a good set of papers, so I re-did the search with different key words]The search for "foundation species" AND "oyster*" yielded 80 results. I filtered these results to select sources where the foundation species model was being applied to understand the function of the oyster reef ecosystem and excluded papers where foundation species was simply used as a description of the oysters. I also removed papers that had already been selected by the "foundation species" search. After filtering this search yielded 16 articles.

\section*{Summary of Foundation Species Model}
\label{sec:found_sp_summary}

The foundation species model was originally described by \citeA{dayton_toward_1972} and has been expanded and developed into a more formalized description of ecosystems that are structured by a strongly interacting and numerically dominant species or assemblage of species \cite{ellison_foundation_2019}. What distinguishes foundation species from other strongly interacting species (e.g., keystone species, ecosystem engineers, etc\ldots) is that foundation species are a dominant member of a community that, through the formation of substantial non--trophic interaction with other species in the community, strongly influence biodiversity, material and energy flows, and the physical and chemical conditions of the system \cite{ellison_foundation_2019}. Foundation species exert their influence through a complex network of interactions that often time can include other foundation species in a ``facilitation cascade'' \cite{angelini_interactions_2011, ellison_foundation_2019, vozzo_cooccuring_2019}. While \citeA{ellison_foundation_2019} provides a formalized definition of what a foundation species is, it is useful to describe the types of effects that foundation species facilitate in ecosystems to determine the unique community they create. 

Although not specifically mentioned in Ellison's formal definition of a foundation species, one of the main things that foundation species do is create biogenic structure that substantially increases the complexity of the system \cite{lenihan_physicalbiological_1999, ellison_loss_2005, ellison_foundation_2019, vozzo_co-occuring_2019, fields_foundation_2022, searles_oyster_2022}. In contrast to other strongly interacting species that either create or modify the structure of the environment and facilitate other organisms (e.g., ecosystem engineers, cornerstone species, etc\ldots), foundation species \emph{dominate} the physical environment with their biogenic structure \cite{ellison_foundation_2019}. In fact \citeA{ellison_foundation_2019} observes that foundation species are often the ``defining'' organisms of an ecosystem type, in that humans use the presence of the foundation species to define the system (e.g., a ``coral'' reef, a ``mangrove'' forest, etc\ldots). Habitat creation can sometimes simply provide physical structure in an otherwise homogeneous environment. \citeA{vozzo_co-occurring_2019} found that the main effect of the mussels (a secondary foundation species who are themselves facilitated by mangroves) is to create habitat for other organisms and \citeA{gedan_accounting_2014} reports that oyster reefs facilitate the recruitment of hooked mussels, who's abundance can exceed the abundance of the oysters themselves. In this sense the foundation species create the physical structure of the system however, through physical--biological coupling \cite{lenihan_physicalbiological_1999}, the structure created by foundation species can alter the physical environment of the system, resulting in an indirect effect on associated organisms \cite{searles_oyster_2022}. 

\citeA{ellison_foundation_2019} highlights the fact that the impact of foundation species is most often via non--trophic interactions. One of the ways that foundation species can facilitate (or suppress) species is through the alteration of the physical or chemical environment of the system. In this case, the dominant presence of the foundation species' biomass alters the physical or chemical environment and creates the conditions required for other species to colonize and persist. Foundation species trees can reduce variation in temperature and moisture both in the soils and in forest streams \cite{ellison_loss_2005}, mussels and surf grass decreased light and temperature variation in tide pools \cite{fields_foundation_2022}, and oyster reefs increase water flow and affect particle settling with increased height \cite{lenihan_physicalbiological_1999}. Kelp forests moderate changes in physical and chemical parameters in the benthos and facilitate higher benthic species richness \cite{lamy_foundation_2020}. Often, the species that are facilitated by the original foundation species can themselves alter the environment facilitating still more species in what \citeA{angelini_interactions_2011} refer to as a ``facilitation--cascade''. In this sense the impact of the foundation species is to create the niche requirements of the species associated with it. 

Through the creation of the dominant habitat structure of the system, typically with their own biomass, foundation species not only create the niche requirements of associated species but also are able to form abundant interspecific interactions \cite{angelini_interactions_2011, ellison_foundation_2019}. These interactions are often through indirect non--trophic effects \cite{ellison_foundation_2019} and foundation species have a much larger effect on ecosystem diversity, structure, and function than even their substantial biomass would predict. Although ecosystems structured by foundation species can often be highly productive \cite{wong_evaluating_2011}, foundation species can often have their effects by reducing energy and material cycling and the variability. For example, hemlock tree stands maintain deep shade in their understory and produce nutrient poor litter, while Chestnut trees sequester nutrients in decay--resistant wood, both of which limit primary production of other species. \cite{ellison_loss_2005}. In aquatic systems, the persistent biomass of foundation species can limit the recruitment of early successional algal species that would otherwise over--grow the system \cite{fields_foundation_2022, CORAL_EXAMPLE}. % keep going with description of foundation species effects...

The foundation species model proposes that an ecosystem contains one or a few numerically dominant species that have a disproportional effect on community structure mainly through non--trophic interactions that typically create complex habitat,  reduce the variation in physical and chemical factors, and reduce material and energy cycling to facilitate and stabilize a specific community \cite{ellison_loss_2005, angelini_interactions_2011, ellison_foundation_2019, fields_foundation_2022}. The non--trophic and interaction--based nature of the effects of foundation species means that many of the observed effects of foundation species are emergent properties. These emergent properties result from the non--linear interactions of the biotic and abiotic elements of the system. For example, \citeA{narwani_interactive_2019} observed that when the two foundation species (water millfoil and mussels) in their shallow pond system co--occurred, they reversed each other's effect on phytoplakton biomass when they were the only species present. Additionally, in the same system the effects of the foundation species on phytoplankton biomass was largely produced by the growth of a single species of cyanobacteria that did not interact strongly with either of the foundation species \cite{narwani_interactive_2019}. Similarly, the observed effect of the stability of the kelp population on the stability of benthic algal and invertebrate population sizes was actually an indirect effect of kelp stability on the species richness of the benthic community and the fact that the more diverse community was more stable \cite{lamy_foundation_2020}. The presence of ``facilitation cascades'' as described by \citeA{angelini_interactions_2011} supports the idea of the effect of emergent properties in foundation species structured systems. Although mangroves are widely considered to be a foundation species, their main effect on biodiversity has been shown to be the result of their facilitation of two other species---rock oysters and a fucalean algae---who then directly affected the biodiversity of the system through non--linear interactive effects \cite{vozzo_co-occuring_2019}.


\subsection*{Oysters as Foundation Species}

Conceptualizing an ecosystem based on the foundation species model, means understanding that system as being structured by the specific facilitation interactions of one or more numerically dominant species \cite{dayton_toward_1972, ellison_loss_2005, angelini_interactions_2011}. Because they are capable of building large biogenic structures that can persist for long time periods, sometimes centuries \cite{lockwood_conservation_2019}, of altering local physical conditions \cite{lenihan_physicalbiological_1999}, and of processing large volumes of materials via filter feeding \cite{newell_ecological_1988} oyster reefs can be conceptualized as foundation species where they occur. The three--dimensional structure of an oyster reef (mainly its height above the otherwise flat sediments) and its heterogeneous complex structure creates, a unique environment that facilitates the growth and persistence of the oysters themselves \cite{lenihan_physicalbiological_1999}, sessile producers and consumers \cite{}, and mobile consumers \cite{smith_restored_2022, searles_oyster_2022}. This facilitation, in part, relies on the way that the reef alters the physical environment by altering flow around the reef to alter the temperature, dissolved oxygen, and other physiochemical factors \cite{lenihan_physicalbiological_1999}. The reef also alters species trophic interactions by providing cover for both predators and prey organisms \cite{smith_restored_2022} and aggregating and increasing resources for consumers \cite{newell_ecological_1988}. As a result of these integrated effects oyster reefs support a unique community of sessile and mobile organisms who's growth and persistence is facilitated by the reef \cite{lenihan_physicalbiological_1999, smith_restored_2022}. 

Consistent with the idea of a foundation species, when the reef structure is destroyed (mostly by destructive harvest methods), then the community becomes less diverse and dominated by non--reef--building organisms. Following the decline of oyster reefs in through the 19\textsuperscript{th} and 20\textsuperscript{th} centuries from overfishing, the habitats previously occupied by oysters became colonized by other benthic filter feeders (e.g., \emph{Corbicula fluminea}) indicating a significant shift in the food web \cite{newell_ecological_1988}. The loss of the reefs also significantly affected material cycling within the esturine system, since the oysters were no longer concentrating phytoplankton biomass into feces and pseudofeces to provided resources to benthic consumers, benthic--pelagic coupling significantly decreased \cite{newell_ecological_1988}. What these observations support is the idea that the effect of the reefs on the system are due to emergent properties that result from the interactions between the oysters and other organisms in a "mature" reef. The function of oyster reefs is dependent on their structural development and complex ecological interactions, indicating that it is the creation of a particular habitat and its associated conditions that drives the effect. For example, \citeA{searles_oyster_2022} found that reef-associated macroinvertebrate communities recovered on the interior of restored reefs but not their margins, suggesting that a certain oyster density and/or reef structure was required to facilitate the colonization of the specific assemblage of  macroinvertebrates that typify a reef community. 

One reason for this effect is likely that the alteration of the physical environment is a key factor in the way that oyster reefs create the specific ecological communities associated with them \cite{lenihan_physicalbiological_1999, searles_oyster_2022}. Because the effects of the reef on the structure and function of the reef community are the result of emergent effects that result from the ``physical--biological coupling'' of the reef and the estuarine environment, the deconstruction of the reef by harvesting practices that destroy the three--dimensional characteristics of the reef, as well as remove individuals will result in the deminishment or elimination of the facultative properties of the reef \cite{lenihan_physicalbiological_1999}. In this way the reef as a foundation species is defined by the emergent properties that arise from the integrated function of the oysters and the other organisms that are facilitated by the effects that the oysters have on the physical--biological coupling of the system \cite{lenihan_physicalbiological_1999, ellison_loss_2005, angelini_interactions_2011}. 

% Transition into a description of the evidence that the ecosystem services model can be applied to oyster reefs

\section*{Summary of Ecosystem Services Model}
\label{sec:ecosys_serv_summary}

In contrast to the foundation species model, the ecosystem services model conceptualizes ecosystems based on the "services" that they provide, in particular services that are valued by humans \cite{}. In this conceptualization, the ecosystem is a source of unaccounted for value to the economic system by enhancing the surplus value of associated the economic system in ways that do not require payment \cite{}. The types and value of ecosystems services for many types of ecosystems have been extensively reviewed \cite{costanza_value_1998}. \citeA{costanza_value_1998} estimated that the economic value of the total ecosystem services of the global biosphere was \$33 trillion USD, which is 1.8 times global GDP at the time that the paper was written. The authors go even further to emphasize that their estimate represents the \emph{minimum} value of ecosystems services provided by the world's ecosystems. \citeA{costanza_value_1998} analysis highlights the enormous contribution that ecosystem services provide for human existence and that it is really impossible to truly account for these services economically. 
More practically the intent of the ecosystem services model is to translate the function of natural ecosystems into market value that can then be used to inform management decisions \cite{}. For example, \citeA{grabowski_economic_2012} found that the economic value the ecosystem services provided by restored oyster reefs exceeded the market value of the oysters harvested from them, indicating that from an economic perspective alone, oyster reefs should be restored and protected from harvest. Used in this way, the ecosystem services model can be used to calculate and communicate the way that the preservation and restoration of ecosystems adds economic value \cite{} and has been used successfully to justify the preservation of EXAMPLE ECOSYSTEMS. 

The ecosystem services model has been applied extensively to oyster reef ecosystems and oyster reefs have been shown to provide shoreline protection, habitat creation for economically valuable fish species, water quality improvement through phytoplankton and excess nutrient removal, and carbon sequestration  \cite{coen_ecosystem_2007, grabowski_economic_2012}. Historically, oyster reefs have been seen as primarily just a source of oyster meat for consumption and shell for construction \cite{}. However, the increased recognition that oyster reefs provide a myriad of valuable ecosystem services has justified the restoration of reefs for more than just subsidizing harvest \cite{}. 

\section*{Application of Models to Oyster Restoration and Aquaculture}
\label{sec:aquaculture_restoration_summary}

\section*{Impact of Models on Understanding Oyster Management}
\label{sec:impact_of_models}

\section*{conclusions}
\label{sec:conclusions}

\bibliographystyle{apacite}
\bibliography{Oyster_Lit_Review}
\end{document}
