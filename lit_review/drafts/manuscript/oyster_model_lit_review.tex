%%%%%%%%%%%%%%%%%%%%%%%%%%%%%%%%%%%%%%%%%%%%%%%%%%
% 
% TITLE: Literature review on the use of models for oyster reefs
% AUTHOR: Kenneth Fortino
% STARTED: 2024-02-12
%
% DESCRIPTION:
%  This is the manuscript for submission to peerJ. 
%  The manuscript is based on the guidelines in the "Literature review template" available at: https://peerj.com/about/author-instructions/
%  The LaTeX is formatted according to the specification in the "LaTeX template via Overleaf" available at: https://peerj.com/about/author-instructions/
%  The LaTeX template was used to create a shell document in the `fortinok` Overleaf projects.
%
% CC-BY 
%
% NOTE: For notes on how I got BibTeX to work, see afternoon reflection from 8 January 2024
%%%%%%%%%%%%%%%%%%%%%%%%%%%%%%%%%%%%%%%%%%%%%%%%%%

\documentclass{article}
\usepackage{apacite}


%\begin{abstract}
%\end{abstract}

\begin{document}

\title{Ecological models of oyster reefs affect conceptualization of reefs and conservation goals.}
\author{Kenneth Fortino}
\date{\today}

\maketitle

\section*{Introduction}
\label{sec:intro}

% Add something about how models are used to conceptualize systems in ecology
Oysters and oyster reefs are conceptualized using both the ecosystem services model and the foundation species model---sometimes in the same paper \cite{mercaldo-allen_oyster_2023}. There is evidence in the literature to support the use of both of these models for understanding oyster ecology but the different models lead to different conceptualizations of the system. 

\section*{Survey Methodology}
\label{sec:surv_method}
% Describe the process by which you ensured that your coverage of the literature was comprehensive and unbiased.

This literature review is intended to address the question of how the foundation species model and the ecosystem services model have been applied to understanding oyster reefs and how the application of these models affect the way that we approach oyster reef conservation, restoration, and aquaculture, particularly in the Chesapeake Bay. To compare the ecosystem services model to the foundation species model, I conducted a Web--of--Science search using the key words "ecosystem services" and "foundation species". The study of foundation species is an active and developing area of research and therefore this broad search yielded 1,136 articles. Since the goal of this search was to develop a comprehensive (but not necessarily exhaustive) understanding of the conceptual structure of the foundation species model, I reviewed the first 100 articles of the search, sorted by "Relevance" in the Web--of--Science results. After reading the title and abstracts of these articles, I eliminated any articles that I determined to be focused on the ecology of specific systems and did not contribute to the broader conceptual understanding of foundation species. Following this culling, I was left with 40 articles. These articles were then augmented with relevant articles that were identified from articles cited from the original search. The final number of articles considered for understanding foundation species was NUMBER.

The same approach was applied to understanding of the conceptual structure of the ecosystem services model. Since the concept of ecosystem services spans more disciplines and has received broader academic attention, this initial search yielded 44,748 articles. Since there is a journal titles "Ecosystem Services", this initial search selected only articles from this journal in the list of most relevant articles. To account for this, I conducted a search that omitted this journal from consideration and then I searched this journal specifically. The search that omitted the Ecosystems Services journal yielded 43,233 articles Since there is a journal titles "Ecosystem Services", this initial search selected only articles from this journal in the list of most relevant articles. To account for this, I conducted a search that omitted this journal from consideration and then I searched this journal specifically. The search that omitted the Ecosystems Services journal yielded 43,233 articles. The search of only the Ecosystem Services journal yielded 1,515 articles.  

\section*{The theoretical framework of the foundation species and ecosystem services concepts}
\label{sec:theoretical_framework}

\subsection{The foundation species concept}
\label{subsec:foundation_species}

The foundation species concept was originally developed by \citeA{dayton_toward_1972} to describe ecological systems where the impact of a numerically dominant species in the system on the abiotic and biotic structure of the structure structures the remaining community.

\section*{Application of the ecological concepts to eastern oyster reef ecosystems}
\label{sec:concept_application}

\section*{The scientific implications of ecological concepts of eastern oyster reef ecosystems}
\label{sec:scientific_implications}

\section*{Impact of Models on Understanding Oyster Management}
\label{sec:impact_of_models}

\section*{Conclusions}
\label{sec:conclusions}

\bibliographystyle{apacite}
\bibliography{foundation_species}
\end{document}
